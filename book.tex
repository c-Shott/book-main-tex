\documentclass[twosided]{book}
\usepackage[nomarginpar,paperheight=16cm, paperwidth=12cm, includehead, nomarginpar, textwidth=10cm, headheight=14pt, ]{geometry}
\usepackage{fourier-orns}
\usepackage{fancyhdr}
\usepackage{subfiles}
\usepackage{lettrine}



\renewcommand{\headrule}{%
\vspace{-8pt}\hrulefill
\raisebox{-2.1pt}{\quad\floweroneleft\decotwo\floweroneright\quad}\hrulefill}
\title{Have You An Educated Heart?}
\author{Gelett Burgess}
\begin{document}
\pagestyle{fancy}
\fancyhf{}
\fancyhead[RE, RO]{\footnotesize Have You An Educated Heart?}
\fancyhead[LE, LO]{\textit{Gelett Burgess}}
\fancyfoot[RE,RO]{\hfill\thepage\hfill}

\maketitle
\lettrine{N}{ow}, Sadie, I knew, was temperamental. Sadie was sensitive. But surely
there wasn’t quite enough in that dull musical comedy, that afternoon,
to make anyone weep. But, as I had noticed Sadie dabbing at her eyes
with her handkerchief, off and on through the first act, when the
curtain went down, I demanded the reason.

\subfile{chapters/chapter1.tex}
\newpage
\lettrine{T}{he} lady was growing a little calmer now. “Why, I’ve called it that to
myself so long that it seems as if anyone ought to understand. Well,
it’s this way. You know you can usually tell an educated man, can’t you?
There’s something about him that’s---oh, I don’t know---extra. Finish, it
is, perhaps. Distinction, or something. He knows how to---”
\subfile{chapters/chapter2.tex}
\newpage

\lettrine{T}{he} Educated Heart! Late that night, alone, I pondered it\ldots\\
midnight, and still it haunted me \ldots the Educated Heart\ldots.
Superkindness, she might have called it. She might have called it Tact.
Vainly, I tried to coin a new word for it. But at the end, Sadie’s
simple term stayed with me$-$the Educated Heart. And so, using that test,
I found myself at length classifying my friends.
\subfile{chapters/chapter3.tex}
\newpage
\lettrine{W}{hy}, there are innumerable ways of insulting the giver of a present.
Listen, for instance, to this acknowledgment, from Zeroine: “All my
life,” she wrote, “I have given away pink azaleas to my friends; and
this is the first time I ever received one.” How’s that for egoism?
Compared to her noble, extravagant and advertised generosity didn’t my
one gift loom pretty small?
\subfile{chapters/chapter4.tex}
\newpage
\lettrine{D}{’you} think it’s easy, this business of giving? Verily, verily, I say
unto you, giving is as much of an art as portrait-painting, or the
making of glass flowers. There really ought to be a four years’ course
in the art of making presents in every college for those who are not
born intelligently generous.
\subfile{chapters/chapter5.tex}
\newpage
\lettrine{K}{ind} you may think you are, and never know you are not kind. And worse
than all the rest are those who force their presents on you.
\subfile{chapters/chapter6.tex}
\newpage
\lettrine{N}{either} do I insist on obedience to custom. You may violate any of the
folk-ways, for all I care. If you refuse to take off your hat to ladies,
I shall think only that you do not feel yourself bound by the elaborate
rules of romantic love concocted by the lazy troubadours and sentimental
chatelaines of the Middle Ages. Offer your left hand to me, if you like,
I care not. We carry no daggers now. I know that most of us would prefer
to dine with a polite murderer than with an honest tinsmith who eats
with his knife. But, by that same token, I know that all such artificial
distinctions are not based upon kindness. They are merely the unwritten
laws of society.
\subfile{chapters/chapter7.tex}
\newpage
\lettrine{Y}{ou} understand, don’t you, that I’m not trying to discuss unkindness, or
even impoliteness? Nor merely legitimate half-kindness do I mean. It
would be absurd to assert that one shouldn’t say, “Here, you may have
this hat$-$I don’t like it.” But wouldn’t it be still more absurd to call
that sort of thing kindness?
\subfile{chapters/chapter8.tex}
\newpage
\lettrine{Y}{ou} call once or twice at the hospital. Do you ever call again? Not
unless you have the Educated Heart. Yet the patient is still perhaps
quite as ill. The plain truth is, if you must know it, most people
really dislike illness. It bores them. It interferes with their
happiness and convenience. It thrusts upon them, too, a disagreeable
burden of sympathy.
\subfile{chapters/chapter9.tex}
\newpage
\lettrine{T}{ruly}, as Sadie said, nothing is so rare as the Educated Heart. And if
you wonder why, just show a kodak group picture$-$a banquet photograph$-$a
photograph of a class. What does every one of us first look at, talk
about? Ourself. And that’s the reason why most hearts are so unlearned
in kindness. Yet none of us likes himself to be forgotten or neglected.
Almost any wife, I verily believe, would prefer actual rudeness to
having a husband pass over her wedding anniversary unnoticed. Even a
blow would prove that she was of some importance in his life.
\subfile{chapters/chapter10.tex}
\newpage
\lettrine{S}{o}, if you are not content with increasing your chest expansion or your
biceps, if you want to enlarge that mystic organ whence flows true
kindness, you must cultivate your imagination, you must learn to put
yourself in another’s place, think his thoughts. There is but one
substitute for imagination, and that is experience. If you have deeply
suffered, perhaps you may have found from your very pain, what real
kindness is. Like Confucius you may have learned politeness from the
impolite. And if you haven’t$-$well, I scarcely think it should be
necessary for you to break a leg or inoculate yourself with the
germs of typhoid. You might do a great deal, really, by exercising
just your common sense.
\subfile{chapters/chapter11.tex}
\newpage
\end{document}
